\section{Purpose of the Tool}

The purpose of this tool is to provide an emulation environment in which
an ISP (or other BGP AS) can analyze the impact of planned deployment of
RPKI validation in local conditions, without impacting their routing
operations. The EOM Tool is expected to assist the operator in scenarios 
such as the following. 

\begin{itemize}
	\item Assessing if the best route selection changes when validation is enabled.
	\item Assessing the manner in which the best path changes when different local pref/weight settings are used.
	\item Combining data from multiple routers in different ASNs (an ISP can have more than one AS) to check for any routing-level inconsistencies.
	\item Assessing potential problems associated with validating prefixes in a multihoming environment.
	\item Assessing whether an ISP's reliance on multiple rpki-rtr manager instances could result in routing-level inconsistencies.
	\item Testing custom prefix assignment and certification conditions for simulation of failures and other special scenarios - e.g. resource transfers.
\end{itemize}

\section{Tool Components}

\Figure{fig/EOM-Block-Diag.png}{Tool Component Block Diagram}{fig:block_diag}

The block diagram above highlights the different sub-components within
the EOM tool. At a high level the EOM tool will fetch RPKI information
from one or more RPKI caches, poll RIB related information from
routers, analyze the different pieces of data retrived from these data
sources against various parameters, and display its results through a
web-based user interface.  

The following sub-sections describe the functionality associated with
each EOM tool sub-component.

\subsection{rpki-rtr-cli}

This component will provide the client interface to the rpki-rtr
protocol.  It will most likely build upon existing rpki-rtr
implementations but instead of communicating with a router this
module will interface with an aggregator module that maintains a
historical record of validated RPKI information.

The rpki-rtr-cli module will have the capability to communicate with
multiple rpki-rtr manager instances, inter-operating with multiple 
rpki-rtr server implementations, in order to be able to mimic cases 
where an ISP relies on multiple rpki-rtr manager instances for serving
validated RPKI information to its various ASBR routers.

\subsection{rtr-status-fetcher}

This module will be responsible for fetching various pieces of status
information from a router, including its routing table and next hop
neighbors. Data from a router will be fetched through RANCID, an open-source
tool that simplifies the automation of routine operations on routers
through expect scripts. The EOM tool may be required to analyze the data
for multiple routers, so the rtr-status-fetcher will also save all fetched
data into a local store through the aggregator module. 

\subsection{aggregator/store}

This module will be responsible for storing and retrieving validated
rpki route objects associated with various rpki-rtr manager instances
and router status information for the different routers queried.
New data will be fed by the rpki-rtr-cli and rtr-status-fetcher
modules and existing data will be queried by other modules that wish to
analyze such data under various operational scenarios. 

\subsection{scenario builder}

The scenario-builder module will be responsible for providing the
central funcionality for enabling a network operator to configure and
develop various types of operational scenarios and router policies to
study the impact of RPKI validation on their network operations. 

This module will interface with the UI module to gather user input on
the types of analyses that are to be performed. It will access previously
saved router and RPKI cache data through the aggregator module and
prepare the data for subsequent analysis by other modules.

In order to build scenarious that include misconfigured or particular
forms of RPKI validation data, the scenario-builder will also interface
with a private RPKI store that will allow the user to build test ROAs on
the fly to simulate various prefix assignment and certification
conditions.

\subsection{route-diff-generator}

This module will provide the first level of analysis over the different
streams of data that are made available by the scenario-builder.
The primary function of this module will be to detect and flag
differences in the routing state for different operational scenarios.

\subsection{diff-reporter}

The diff-reporter module will take the raw results generated by the
route-diff-generator module and will transform that data to a form that
is useful for user consumption. Differences will initially be represented as
simple text but will later be color coded and marked up to clearly
highlight any differences. 

\subsection{analyzer}

The analyzer module will operate on data generated by the
route-diff-generator to make inferences about potential routing level
inconsistencies that may be useful to flag to the user. 

\subsection{notifier}

The notifier module will be responsible for processing the results from 
the analyzer module in order to present those results to the user in an
intuitive manner in order to help them develop additional test scenarios
in an iterative manner. In the future, when the EOM tool is used as a
monitoring tool, this module will also provide user notification
capabilities to warn and alert the user of existing and impending
problems.

\subsection{UI}

The User Interface module will provide the engine for displaying the
various pieces of data generated by the data-reporter and notifier
modules. This module will also provide the configuration interface for
the users, to enable them to specify the location and parameters
associated with the different routers and RPKI stores, and to enable them
to define the parameters associated with their different scenarios of
interest.


\section{Development Phases Overview}

This section outlines the likely progression of the various tool
capabilities.

\subsection{Phase 1}

In the initial phase we will add basic support for the
rpki-rtr-cli, rtr-status-fetcher, and route-diff-generator modules so
that the tool provides the ability to analyze routing information
extracted from a border router and produce a one time text report of
route validity.

The aggregator, analyzer, notifier and scenario-builder modules will
be stubbed out and will provide only basic API compatibiity. 

\subsection{Phase 2}

In the second phase we will extend rtr-status-fetcher to periodically poll
multiple routers for their RIB information, and will also add initial
capability to the aggregator, route-diff-generator and diff-reporter modules
to produce enhanced reporting capabilities through a web-based
front-end.

We will also add initial capability to the scenario builder to
help the operator assess the effects of RPKI validation under 
different router policy (e.g. local-pref) settings. 

\subsection{Phase 3}

In this phase we will enhance the EOM tool to provide post-deployment
monitoring support in addition to supporting pre-deployment planning. 

The scenario-builder component will be enhanced to support the inclusion
of custom RPKI data fed through a private RPKI store, which will be
anchored to a private Trust Anchor. 

The analyzer component will be enhanced to automate the analysis of
various pieces of data in order to provide the operator with the
capability to monitor the consistency of their RPKI and routing data on
an ongoing basis. 

The nature of analysis changes once RPKI validation has been enabled
within an operational environment. Here the EOM tool must make a
determination about how the absence (rather than the presence) of
validation is likely to impact best path selection.
Since routers may not have the capability to display their RPKI
state even when they actually use RPKI information in their best path
selection, inferring the best path in the absence of validation may
be non-trivial. The specifics of this analysis will need to be
revisted in due course. 

%\section{Module Interfaces}
%
%This section describes the currently proposed API for the various
%sub-modules associated with the EOM tool. Note that the API is in a
%state of flux and may change (sometimes, radically) as development of
%the tool components proceeds.
%
%TODO
%
%\subsection{rpki-rtr-cli}
%\subsection{rtr-status-fetcher}
%\subsection{aggregator/store}
%\subsection{scenario builder}
%\subsection{route-diff-generator}
%\subsection{diff-reporter}
%\subsection{analyzer}
%\subsection{notifier}
%\subsection{UI}



\section{Synergy with other Efforts}

A number of implementations exist for the server and client
components of the rpki-rtr protocol. The client component of the
rpki-rtr protocol is usually integrated into router implementations.
However, currently, there is no tool that allows the user to compare the
data from routers juxtaposed with RPKI validation data in order to study
the effects of RPKI validation on operational routing. The EOM tool
fills this gap. However, where possible, the EOM tool will interoperate
with and build upon existing implementations of the rpki-rtr protocol
in order to minimize duplication of effort.

The EOM tool will also build upon other route collection efforts. For
example, the tool is envisioned to interact with multiple rpki-rtr
servers, some of which may be external to the ISP/network operation
where the EOM analysis is being performed. Likewise, the EOM tool will
also fetch data from various public looking glasses in order to provide
external validity of the routing information being analyzed. 

Finally, the EOM tool components will be open sourced in order to
encourage collaborative participation by the operator/user community and
to widen the use of the tool as a deployment aide.
